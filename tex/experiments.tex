\chapter{Метод Якоби}

\section{Эксперимент 1.}

Влияние выбора $\varepsilon$ на количество итерационных шагов и максимальную погрешность между точным и приближенным решением.

В этом эксперименте мы зафиксировали $n = 10$, вычислили $h=0.1$, выбрали $\varepsilon = 10^{-1}, 10^{-2}, \dots, 10^{-6}$. Наша задача вычислить точное решение $u$, вычислить приближенное решение $y$ при разных $\varepsilon$, выяснить на каком итерационном шаге $k$  остановился процесс, найти максимальную погрешность вычислений $\Delta$ при каждом выбранном $\varepsilon$.

\input{jacobi_epsilon__gen.tex}

\section{Эксперимент 2.}

Влияние выбора количества узлов $n$ на количество итерационных шагов и максимальную погрешность метода.

В данном эксперименте мы зафиксировали $\varepsilon = 10^{-4}$. Наша задача найти количество итерационных шагов $k$ и максимальную погрешность вычислений $\Delta$ при каждом $n$.

\input{jacobi_dimension__gen.tex}

\chapter{Метод Зейделя}

\section{Эксперимент 1.}

Влияние выбора $\varepsilon$ на количество итерационных шагов и максимальную погрешность между точным и приближенным решением.

В этом эксперименте мы зафиксировали $n = 10$, вычислили $h=0.1$, выбрали $\varepsilon = 10^{-1}, 10^{-2}, \dots, 10^{-6}$. Наша задача вычислить точное решение $u$, вычислить приближенное решение $y$ при разных $\varepsilon$, выяснить на каком итерационном шаге $k$  остановился процесс, найти максимальную погрешность вычислений $\Delta$ при каждом выбранном $\varepsilon$.

\input{seidel_epsilon__gen.tex}

\section{Эксперимент 2.}

Влияние выбора количества узлов $n$ на количество итерационных шагов и максимальную погрешность метода.

В данном эксперименте мы зафиксировали $\varepsilon = 10^{-4}$. Наша задача найти количество итерационных шагов $k$ и максимальную погрешность вычислений $\Delta$ при каждом $n$.

\input{seidel_dimension__gen.tex}

\chapter{Метод верхней релаксации}

\section{Эксперимент 1.}

Влияние выбора $\varepsilon$ на количество итерационных шагов и максимальную погрешность между точным и приближенным решением.

В этом эксперименте мы зафиксировали $n = 10$, вычислили $h=0.1$, выбрали $\varepsilon = 10^{-1}, 10^{-2}, \dots, 10^{-6}$. Наша задача вычислить точное решение $u$, вычислить приближенное решение $y$ при разных $\varepsilon$, выяснить на каком итерационном шаге $k$  остановился процесс, найти максимальную погрешность вычислений $\Delta$ при каждом выбранном $\varepsilon$.

\input{relaxation_epsilon__gen.tex}

\section{Эксперимент 2.}

Влияние выбора количества узлов $n$ на количество итерационных шагов и максимальную погрешность метода.

В данном эксперименте мы зафиксировали $\varepsilon = 10^{-4}$. Наша задача найти количество итерационных шагов $k$ и максимальную погрешность вычислений $\Delta$ при каждом $n$.

\input{relaxation_dimension__gen.tex}

\section{Эксперимент 3.}

Влияние выбора параметра $\omega$ на количество итерационных шагов и максимальную погрешность метода.

В данном эксперименте мы зафиксировали $\varepsilon = 10^{-4}$. Наша задача найти количество итерационных шагов $k$ и максимальную погрешность вычислений $\Delta$ при каждом $\omega$.

\input{omega__gen.tex}

\chapter{Метод Прогонки}

\section{Влияние выбора количества узлов $n$ на количество итерационных шагов и максимальную погрешность метода.}

В данном эксперименте мы зафиксировали $\varepsilon = 10^{-4}$. Наша задача найти количество итерационных шагов $k$ и максимальную погрешность вычислений $\Delta$ при каждом $n$.

\input{thomas_dimension__gen.tex}