\section{Расчетные формулы исследуемых методов}
Критерий остановки во всех методах мы выбрали следующий:
\begin{equation}
	\left| y^{k+1} - y^k \right| \leq \varepsilon
\end{equation}

Начальное приближение для всех методов было выбрано одинаковое -- единичный вектор.

\paragraph{Метод Якоби}
Данный итерационный метод определяет $k+1$ приближение решения системы формулой:
\begin{equation}
	y^{k+1}_i = \frac{1}{a_{i i}}\left(f_i - \sum^n_{j=1, j \neq i} a_{i j}y^k_j \right).
\end{equation}

\paragraph{Метод Зейделя}
Основная идея метода заключается в том, что при вычислении $k+1$-го приближения решения $y_i$ учитываются уже вычисленные ранее компоненты $k+1$-го приближения решения.
\begin{equation}
	y^{k+1}_i = \frac{1}{a_{i i}}\left(f_i - \sum^{i-1}_{j=1} a_{i j}y^{k+1}_j - \sum^n_{j=i+1} a_{i j}y^k_j \right).
\end{equation}

\paragraph{Метод верхней релаксации}
Данный метод является модификацией метода Зейделя. Главную роль играет параметр $\omega \in (1, 2)$, выбирающийся таким образом, чтобы на каждом шаге итерационного процесса уменьшалась величина, характеризующая близость полученного решения к искомому решению системы. Расчетная формула метода имеет вид:
\begin{equation}
	y^{k+1}_i = (1-\omega)y^k_i+\frac{\omega}{a_{i i}}\left(f_i - \sum^{i-1}_{j=1} a_{i j}y^{k+1}_j - \sum^n_{j=i+1} a_{i j}y^k_j \right).
\end{equation}

\paragraph{Метод прогонки}
Данный метод используется, когда матрица $A$ является трехдиагональной матрицей. Метод основан на принципе последовательного исключения неизвестных. Метод будем реализовывать с помощью следующих формул
\begin{equation}
	\begin{array}{rcl}
		\alpha_{i+1}&=&\frac{-a_{i, i+1}}{a_{i, i}+\alpha_i a_{i, i-1}}\\
		\beta_{i+1}&=&\frac{-a_{i, i-1}\beta_i+f_i}{a_{i, i}+\alpha_i a_{i, i-1}}\\
	\end{array}
\end{equation}
\begin{equation}
	y_i=\alpha_{i+1}y_{i+1} + \beta_{i+1}
\end{equation}
\begin{equation}
	y_{n-1}=\frac{f_{n-1} - \beta_{n-1}a_{n-2, n-1}}{a_{n-1, n-1}+\alpha_{n-1}a_{n-2, n-1}}
\end{equation}