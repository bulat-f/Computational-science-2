\chapter{Введение}
\section{Постановка задачи и исходные данные}
Дана система линейных алгебраических уравнений (СЛАУ):
\begin{equation}\label{eq:syst}
	\left\{
	\begin{array}{rcl}
		-a_{i+1}(y_{i+1} - y_i) + a_i(y_i - y_{i-1}) + g_i h^2 y_i&=&f_i h^2, i = 1, n-1\\
		y_0 = y_n&=&0,\\
	\end{array}
	\right.
\end{equation}
где $n$ - это количество узлов, а $h = \frac{1}{n}$, $a_i = p(ih)$, $g_i = g(ih)$, $f_i = f(ih)$.

Соответсвующие функции определены следующим образом
\begin{equation}\label{ux}
	u(x) = x^3(1-x)^4
\end{equation}
\begin{equation}\label{px}
	p(x) = 1+x^3
\end{equation}
\begin{equation}\label{gx}
	g(x) = 1+x
\end{equation}
\begin{equation}\label{fx}
	f(x) = -(pu')'+gu
\end{equation}

Цель задания:
\begin{enumerate}
	\item Решить СЛАУ итерационными методами:
		\begin{itemize}
			\item Якоби
			\item Зейделя
			\item верхней релаксации
		\end{itemize}
	\item Решить СЛАУ методом прогонки
	\item Провести численные эксперименты
	\item Сравнить методы и сделать выводы
\end{enumerate}

\section{Предварительные вычисления}

Исследуемые нами методы предназначены для решения систем алгебраических уравнений вида $Ay=f$, где A -- квадратная матрица, y -- вектор, являющийся решением системы, f -- вектор правой части.
В нашей задаче $y_0 = y_n = 0$,  поэтому матрица $A$ выглядит следующим образом:

$$
	A = 
	\left(
	\begin{array}{ccccccc}
		a_{1 1}&a_{1 2}&0&\dots&0&0&0\\
		a_{2 1}&a_{2 2}&a_{2 3}&\dots&0&0&0\\
		\dots&\dots&\dots&\dots&\dots&\dots&\dots\\
		0&0&0&\dots&a_{n-2, n-3}&a_{n-2, n-2}&a_{n-2, n-1}\\
		0&0&0&\dots&0&a_{n-1, n-2}&a_{n-1, n-1}\\
	\end{array}
	\right).
$$

Компоненты матрицы $A$ вычисляются по следующей формуле.

\begin{equation}
	\begin{array}{rcl}
		a_{i+1,i}&=&-a_i,\\
		a_{i, i}&=&a_i+a_{i+1}+g_ih^2,\\
		a_{i, i+1}&=&-a_{i+1},\\
	\end{array}
\end{equation}
где $a_i$, $g_i$ и $h$ опрделены выше.

Из (\ref{ux}), (\ref{px}), (\ref{gx}) и (\ref{fx}) найдем формулу, по которой вычисляется вектор правой части: $$f(x) = -3x^4(x-1)^3(7x-3)-6x(1+x^3)(x-1)^2(7x^2-6x+1)+(1+x)x^3(1-x)^4$$

\section{Расчетные формулы исследуемых методов}
Критерий остановки во всех методах мы выбрали следующий:
\begin{equation}
	\left| y^{k+1} - y^k \right| \leq \varepsilon
\end{equation}

Начальное приближение для всех методов было выбрано одинаковое -- единичный вектор.

\paragraph{Метод Якоби}
Данный итерационный метод определяет $k+1$ приближение решения системы формулой:
\begin{equation}
	y^{k+1}_i = \frac{1}{a_{i i}}\left(f_i - \sum^n_{j=1, j \neq i} a_{i j}y^k_j \right).
\end{equation}

\paragraph{Метод Зейделя}
Основная идея метода заключается в том, что при вычислении $k+1$-го приближения решения $y_i$ учитываются уже вычисленные ранее компоненты $k+1$-го приближения решения.
\begin{equation}
	y^{k+1}_i = \frac{1}{a_{i i}}\left(f_i - \sum^{i-1}_{j=1} a_{i j}y^{k+1}_j - \sum^n_{j=i+1} a_{i j}y^k_j \right).
\end{equation}

\paragraph{Метод верхней релаксации}
Данный метод является модификацией метода Зейделя. Главную роль играет параметр $\omega \in (1, 2)$, выбирающийся таким образом, чтобы на каждом шаге итерационного процесса уменьшалась величина, характеризующая близость полученного решения к искомому решению системы. Расчетная формула метода имеет вид:
\begin{equation}
	y^{k+1}_i = (1-\omega)y^k_i+\frac{\omega}{a_{i i}}\left(f_i - \sum^{i-1}_{j=1} a_{i j}y^{k+1}_j - \sum^n_{j=i+1} a_{i j}y^k_j \right).
\end{equation}

\paragraph{Метод прогонки}
Данный метод используется, когда матрица $A$ является трехдиагональной матрицей. Метод основан на принципе последовательного исключения неизвестных. Метод будем реализовывать с помощью следующих формул
\begin{equation}
	\begin{array}{rcl}
		\alpha_{i+1}&=&\frac{-a_{i, i+1}}{a_{i, i}+\alpha_i a_{i, i-1}}\\
		\beta_{i+1}&=&\frac{-a_{i, i-1}\beta_i+f_i}{a_{i, i}+\alpha_i a_{i, i-1}}\\
	\end{array}
\end{equation}
\begin{equation}
	y_i=\alpha_{i+1}y_{i+1} + \beta_{i+1}
\end{equation}
\begin{equation}
	y_{n-1}=\frac{f_{n-1} - \beta_{n-1}a_{n-2, n-1}}{a_{n-1, n-1}+\alpha_{n-1}a_{n-2, n-1}}
\end{equation}