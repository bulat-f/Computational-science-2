\section{Эксперементы}

\paragraph{Метод Якоби}

\subparagraph{Эксперимент 1.}

Влияние выбора $\varepsilon$ на количество итерационных шагов и максимальную погрешность между точным и приближенным решением.

В этом эксперименте мы зафиксировали $n = 10$, вычислили $h=0.1$, выбрали $\varepsilon = 10^{-1}, 10^{-2}, \dots, 10^{-6}$. Наша задача вычислить точное решение $u$, вычислить приближенное решение $y$ при разных $\varepsilon$, выяснить на каком итерационном шаге $k$  остановился процесс, найти максимальную погрешность вычислений $\Delta$ при каждом выбранном $\varepsilon$.

Решение.

\begin{tabular}{|c|cccccc|}
\hline
$\varepsilon$&$10^{-1}$&$10^{-2}$&$10^{-3}$&$10^{-4}$&$10^{-5}$&$10^{-6}$\\
\hline
$\delta$&$0.00368073
$&$0.00368073
$&$0.00198952
$&$0.000419443
$&$4.09568e-05
$&$4.43895e-06
$\\
\hline
$k$&$1
$&$1
$&$4
$&$18
$&$42
$&$65
$\\
\hline
\end{tabular}

\includegraphics{jacobi_iterations.png}

\paragraph{Метод Зейделя}

\subparagraph{Эксперимент 1.}

Влияние выбора $\varepsilon$ на количество итерационных шагов и максимальную погрешность между точным и приближенным решением.

В этом эксперименте мы зафиксировали $n = 10$, вычислили $h=0.1$, выбрали $\varepsilon = 10^{-1}, 10^{-2}, \dots, 10^{-6}$. Наша задача вычислить точное решение $u$, вычислить приближенное решение $y$ при разных $\varepsilon$, выяснить на каком итерационном шаге $k$  остановился процесс, найти максимальную погрешность вычислений $\Delta$ при каждом выбранном $\varepsilon$.

Решение.

\begin{tabular}{|c|cccccc|}
\hline
$\varepsilon$&$10^{-1}$&$10^{-2}$&$10^{-3}$&$10^{-4}$&$10^{-5}$&$10^{-6}$\\
\hline
$\delta$&$0.00362114
$&$0.00362114
$&$0.00176776
$&$0.000220994
$&$2.12186e-05
$&$2.02837e-06
$\\
\hline
$k$&$1
$&$1
$&$4
$&$15
$&$27
$&$39
$\\
\hline
\end{tabular}

\includegraphics{seidel_iterations.png}

\paragraph{Метод верхней релаксации}

\subparagraph{Эксперимент 1.}

Влияние выбора $\varepsilon$ на количество итерационных шагов и максимальную погрешность между точным и приближенным решением.

В этом эксперименте мы зафиксировали $n = 10$, вычислили $h=0.1$, выбрали $\varepsilon = 10^{-1}, 10^{-2}, \dots, 10^{-6}$. Наша задача вычислить точное решение $u$, вычислить приближенное решение $y$ при разных $\varepsilon$, выяснить на каком итерационном шаге $k$  остановился процесс, найти максимальную погрешность вычислений $\Delta$ при каждом выбранном $\varepsilon$.

Решение.

\begin{tabular}{|c|cccccc|}
\hline
$\varepsilon$&$10^{-1}$&$10^{-2}$&$10^{-3}$&$10^{-4}$&$10^{-5}$&$10^{-6}$\\
\hline
$\delta$&$0.0038299
$&$0.0038299
$&$0.000158805
$&$1.18703e-05
$&$2.18802e-06
$&$2.96632e-07
$\\
\hline
$k$&$1
$&$1
$&$6
$&$10
$&$12
$&$16
$\\
\hline
\end{tabular}

\includegraphics{relaxation_iterations.png}