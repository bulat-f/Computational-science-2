\section{Предварительные вычисления}

Исследуемые нами методы предназначены для решения систем алгебраических уравнений вида $Ay=f$, где A -- квадратная матрица, y -- вектор, являющийся решением системы, f -- вектор правой части.
В нашей задаче $y_0 = y_n = 0$,  поэтому матрица $A$ выглядит следующим образом:

$$
	A = 
	\left(
	\begin{array}{ccccccc}
		a_{1 1}&a_{1 2}&0&\dots&0&0&0\\
		a_{2 1}&a_{2 2}&a_{2 3}&\dots&0&0&0\\
		\dots&\dots&\dots&\dots&\dots&\dots&\dots\\
		0&0&0&\dots&a_{n-2, n-3}&a_{n-2, n-2}&a_{n-2, n-1}\\
		0&0&0&\dots&0&a_{n-1, n-2}&a_{n-1, n-1}\\
	\end{array}
	\right).
$$

Компоненты матрицы $A$ вычисляются по следующей формуле.

\begin{equation}
	\begin{array}{rcl}
		a_{i,i-1}&=&-a_i,\\
		a_{i, i}&=&a_i+a_{i+1}+g_ih^2,\\
		a_{i, i+1}&=&-a_{i+1},\\
	\end{array}
\end{equation}
где $a_i$, $g_i$ и $h$ опрделены выше.

Из (\ref{ux}), (\ref{px}), (\ref{gx}) и (\ref{fx}) найдем формулу, по которой вычисляется вектор правой части: $$f(x) = -3x^4(x-1)^3(7x-3)-6x(1+x^3)(x-1)^2(7x^2-6x+1)+(1+x)x^3(1-x)^4$$